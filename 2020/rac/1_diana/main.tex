\Title{Analiza uticaja prosečnog vremena izvršavanja i broja otvorenih datoteka programa na prioritet i vreme izvršavanje procesa u Linuksu}
\TitleEng{Analysis of the impact of average execution time and the number of open files of a program on priority and time of execution of the processes in Linux}
\Author{Diana Šantavec}
\begin{Abstract}
Razmatrane su dve modifikacije Linuksovog O(1) planera procesa (verzija jezgra: 2.6.22) da bi se ispitalo kakav uticaj na prioritet i vreme izvršavanja procesa imaju prosečno vreme izvršavanja i učestalost pokretanja nekog programa, kao i broj otvorenih datoteka. Prva modifikacija je u heš tabeli pamtila prosečno vreme izvršavanja i broj pokretanja izvršnog koda, određivanjem njegove heš vrednosti. Postignuto je duže vreme izvršavanja učitavanja procesa, ali su na odrađenim testovima, koji su imitirali procese koji većinu svog vremena provode čekajući poziv, dobijeni rezultati bolji nego sa originalnim O(1) algoritmom, a slični onima sa CFS algoritmom. U testovima koji su predstavljali procese čije izvršavanje iziskuje puno procesorskih resursa, modifikovanim algoritmom su postignuti bolji rezultati nego CFS algoritmom, a slični ili gori nego originalnim O(1) algoritmom. Druga modifikacija je brojala otvorene datoteke svakog procesa i na osnovu njihovog broja dodeljivala bonuse procesima. U ovom slučaju rezulati su pokazali da je vreme izvršavanja samog planera bilo produženo, ali je omogućeno da procesi sa više otvorenih datoteka dobiju viši prioritet.
\end{Abstract}
\begin{AuthorEx}
Diana Šantavec (2001), Vrbas, učenica 4. razreda elektrotehničke škole „Mihajlo Pupin” u Novom Sadu

MENTOR:\\
Stefan Nožinić, diplomirani inženjer elektrotehnike i računarstva, inženjer za razvoj softvera, Continental Automotive d.o.o.
\end{AuthorEx}
\begin{AbstractEng}
Process scheduling in multiprogramming operating systems is a problem on which the performance of the entire system depends. The idea behind this project was analyzing the impact of execution time on CPU and the number of executions of a program, and analyzing the impact of the number of opened files on the priority of a process.

Modification of scheduling algorithm, so the number of execution times and execution time on CPU have an impact, is implemented by the introduction of a hash table (where the key represents the hash value of the executable code). Execution time on the CPU was read from the process descriptor and was saved as the average value in the hash table together with the number of executions. Depending on the ratio of execution time on CPU, and the number of execution times, the process has got bonuses/penalties in the range from -5 to 5. This enabled greater priority for processes that have shorter execution time and vice versa. For done tests, this algorithm gave better results than the CFS algorithm, but worse results than the original O(1) algorithm for the CPU bound processes. For tests with processes that imitated daemon processes, results of the modified algorithm were better then the original O(1) algorithm when there were more than one active “daemon” process. The results compared to the CFS algorithm were similar or worse.

Another implementation was made by modification of part of the code in which are counted bonuses/penalties. There was added file counting for the process every five seconds. In this case, only bonuses are added in the range from 1 to 5 (first bonus is added when there are 5 open files because each process has 3 files opened by default (standard streams)).

The first modification slowed down process loading, and the other scheduling algorithm itself, but both gave some better results for certain cases. Depending on system needs, both implemented modifications could have usage, or even their combination.
\end{AbstractEng}

\StartDoublePaper
\label{rac.diana}

\section{Uvod}

U toku rada računara pokrene se više procesa, međutim, procesorsko jezgro može da izvršava samo jedan proces u datom trenutku. Iz tog razloga procesi moraju, ili u potpunosti završiti svoj rad, ili prepustiti procesor drugom procesu, kako bi korisnik imao utisak paralelnog izvršavanja procesa (Tanenbaum i Bos 2015: 86). Smenom procesa postiže se i bolja iskorišćenost procesora, pošto procesor ne mora da čeka proces dok isti izvršava neku ulazno-izlaznu operaciju (Marić 2017: 50).

Da bi se procesi mogli smenjivati, neophodno je zapamtiti trenutno stanje procesa u trenutku kada mu se oduzmu procesorski resursi, što se postiže uvođenjem kontrolnog bloka procesa.

Kontrolni blok procesa je struktura podataka koja pored trenutnog stanja procesa sadrži i druge podatke o procesu, koji se razlikuju u različitim implementacijama operativnih sistema, kao što su: prioritet, status, jedinstveni identifikator (PID), naziv korisnika koji je pokrenuo proces, zauzeti ulazno-izlazni uređaji, i slično (Tanenbaum i Bos 2015: 94).

\AuthorExHere

Kako kontrola prepuštanja procesorskih resursa iz samog programa otežava programiranje i dovodi do slabijih performansi operativnog sistema (gledajući sisteme koji se danas koriste), uveden je planer procesa (\emph{process scheduling algorithm}, u daljem tekstu: planer). Planer je algoritam koji vodi računa o rasporedu i dužini izvršavanja procesa u operativnom sistemu. S obzirom na različite upotrebe operativnih sistema i različite tipove procesa (\emph{I/O bound} - procesi koji veći deo izvšavanja provode čekajući ulazno/izlazne uređaje i \emph{CPU bound} - procesi koji u toku izvršavanja intenzivno koriste procesorske resurse), postoje i različiti algoritmi za planere, koji obezbeđuju različite performanse operativnog sistema (Šantavec 2019).

Ovaj rad se zasniva na modifikaciji Linuksovog O(1) planera. Analizirani su vreme i učestalost izvršavanja procesa kada su u određivanje njegovog prioriteta bili uključeni prosečno vreme izvršavanja ili broj otvorenih datoteka. Implementacija algoritama je izvršena postepeno, tako da su posebno rađeni testovi za ispitivanje uticaja prosečnog vremena izvršavanja, a posebno testovi za ispitivanje uticaja broja otvorenih datoteka. Rezultati su upoređeni, kako međusobno, tako i sa originalnim O(1) planerom. Takođe su upoređeni i sa CFS planerom, koji se koristi i danas, u verziji jezgra 2.6.23.

\section{Metode}

U ovoj sekciji su objašnjeni pojmovi korišćeni za implementaciju projekta u odgovarajućim verzijama.

\subsection{Proces}

Svaki proces ima kontrolni blok procesa (u daljem tekstu: kontrolni blok), u kojem se pamti stanje nakon što mu se oduzmu procesorski resursi. Kako je modifikovan O(1) algoritam, posmatran je samo njegov kontrolni blok (verzija 2.6.22).  Ovaj kontrolni blok čini struktura \texttt{task\_struct} i elementi bitni za ovo istraživanje su:

\begin{itemize}
    \item \texttt{prio} (označena celobrojna vrednost) -- dinamički prioritet procesa u intervalu od 0 do 139, na njega utiču bonusi/penali iz planera
    \item \texttt{static\_prio} (označena celobrojna vrednost) - prioritet koji je zadao korisnik poznat i kao nice vrednost. Ne menja se, osim ukoliko korisnik to ne zahteva i njegova vrednost se kreće od -20 do 19 (-20 je najveći, a 19 najmanji prioritet)
    \item \texttt{sched\_time} (neoznačena celobrojna vrednost veličine najmanje 64 bita) -- vreme koje je proces proveo koristeći procesor
    \item \texttt{pid} (označena celobrojna vrednost tipa \texttt{pid\_t}) -- PID (process identification), jedinstveni identifikator (Tanenbaum i Bos 2015: 734)
    \item \texttt{files} -- tabela otvorenih datoteka sa jedinstvenim identifikatorom za svaku.
\end{itemize}

\subsection{O(1) planer procesa}

Kako se rad zasniva na modifikaciji O(1) planera, sledi opis njegovog načina rada. Naziv je dobio po tome što vreme za odabir procesa, kom treba da se dodele procesorski resursi, ne zavisi od broja procesa, stoga ima algoritamsku kompleksnost O(1). (Tanenbaum i Bos 2015: 749). Svakom procesoru je dodeljen red opsluživanja (\emph{queue}). Red opsluživanja sa kog se pokreću procesi sadrži dva niza, jedan sa procesima koji čekaju na procesorske resurse i jedan sa procesima koji su potrosili svoje procesorske resurse. Procesi se pokreću redom po prioritetu iz niza koji sadrži procese koji nisu iskoristili procesorske resurse. Kada proces potroši procesorsko vreme koje mu je dodeljeno (ili sam prepusti procesor) prebacuje se u niz sa procesima koji su iskoristili svoje procesorsko vreme, i preračunavaju mu se prioritet i kvanta procesorskog vremena koja mu se dodeljuje (vreme koje proces dobije na procesoru između dva prekida).

Na sam prioritet procesa utiče koliko on koristi procesorskih resursa i na osnovu toga mu se daju bonusi/penali od -5 do +5. Procesima koji su posebno interaktivni može da se dopusti još jedno izvršavanje. Kvanta procesorskog vremena koju proces može da dobije je unapred mapirana za svaki statički prioritet (\emph{nice} vrednost). Za najveći prioritet (-20) ona iznosi 800 ms, za najmanji prioritet (+19), iznosi 5 ms. Kada se niz sa procesima koji čekaju procesorkso vreme isprazni, dolazi do smene pokazivača, tako da niz sa procesima koji su potrošili svoje procesorsko vreme postaje niz sa procesima koji čekaju procesorsko vreme i obrnuto (Love 2005: The Linux Scheduling Algorithm; Ishkov 2015).

\subsection{CFS planer procesa}

CFS (Completely Fair Scheduler) algoritam je zamenio O(1) algoritam i koristi se i danas za procese koji se ne izvršavaju u realnom vremenu.

Algoritam je zasnovan na konceptu modela planera za idealan procesor koji može da izvršava više procesa istovremeno (\emph{multitasking procesor}). CFS algoritam ovom konceptu teži tako što za svaki proces preračunava vreme koje će dobiti, uzimajući u obzir i ostale procese.

Red opsluživanja sa kog se pokreću procesi nije izdeljen na nizove kao u O(1) algoritmu, već predstavlja crveno-crno stablo (Cormen, Leiserson, Rivest, Stein 2009: 308-311). Procesi se sortiraju u crveno-crnom stablu po vremenu koje su proveli na procesoru (izraženo u nanosekundama). Vreme koje proces provodi na procesoru ne protiče jednako za sve procese, već zavisi od njihovog dinamičkog prioriteta (\emph{prio} vrednost) i statičkog prioriteta (\emph{nice} vrednosti). Za procese nižeg prioriteta, vreme brže protiče, i pre će ostati bez procesorskih resursa od procesa sa višim prioritetom. Kada u stablu postoji proces sa manjim vremenom izvršavanja (sa uračunatim bonusima i penalima) od procesa koji se trenutno pokreće, trenutno aktivan proces će biti premešten u stablu i prepustiće procesorske resurse sledećem procesu. Sledeći proces koji se pokreće je uvek proces koji se nalazi skroz levo u stablu (Tanenbaum i Bos 2015: 749).

\subsection{Heš funkcija i heš tabela}

Za prvu predloženu modifikaciju je potrebno uvesti pojmove heš funkcije i heš tabele, kao i korišćen algoritam za dobijanje heš vrednosti.

Heš funkcija je funkcija koja dati niz karaktera proizvoljne dužine mapira na niz karaktera fiksne duzine. To su jednosmerne funkcije, što znači da se iz dobijene vrednosti ne može dobiti originalna vrednost, a da se ne isproba svaka moguća kombinacija (Cormen et al. 2015: 262).

Heš tabela omogućava brz pristup podacima. Podaci su sortirani u niz i svaki podatak poseduje svoj ključ. Ključ se dobija kao ostatak pri deljenju vrednosti heš funkcije koju taj podatak poseduje i broja elemenata niza. U slučaju da dva podatka dobiju istu vrednost ključa, dolazi do kolizije koju je moguće rešiti na raličite načine, ali ovi slučajevi nisu razmatrani u modifikacijama.

Za određivnje ključa za heš tabelu, u ovom radu je korišćena djb2 heš funkcija (Algorithms in C 2021: Hash algorithms). Heš se dobija algoritmom:
\[
    \verb|hash(i) = hash(i-1) * 33^str[i]|
\]
gde je: \texttt{hash} -- rezultujuća heš vrednost (inicijalizovana na 5381), \texttt{str} -- string koji se hešuje, a
33 -- konstanta.

Na kraju se uzima ostatak pri deljenju dobijenog heša sa 100,003 kao ključ za heš tabelu. Ideja je bila da se implementira tabela sa 100,000 ulaza, međutim, kako je poželjno da broj bude prost, odabarna je ova vrednost (Knuth 1998: 515 i 516).

\subsection{Datoteke i njeni deskriptori}

Druga modifikacija broji otvorene datoteke svakog procesa. Iz tog razloga u nastavku se nalazi kratko objašnjenje za datoteke i njene deskriptore u Linuksu. 

Procesi imaju mogućnost da otvaraju datoteke da bi pročitali podatke ili upisali nešto. Kako se datotekama predstavlja i komunikacija između procesa, komunikacija sa ulazno/izlaznim uređajima itd., bitno je da proces što pre završi rad sa datotekom i da je oslobodi za neki drugi proces. Iz ovog razloga je ideja da se za svaki proces broje otvorene datoteke i da taj broj utiče na prioritet procesa.

Struktura podataka koja omogućava rad sa datotekama se naziva deskriptor datoteka. Broj otvorenih datoteka ne mora biti jednak broju ovorenih deskriptora datoteka (Linux Interview Questions 2020). Svaki proces ima dodeljena bar tri deskriptora datoteka: stdin (standardni ulaz), stdout (standardni izlaz), stderr (standardni ispis za greške) (Linux Programmer's manual: stdin(03) 2021).


\section{Implementacija algoritma}

Ova sekcija opisuje implementacije predloženih izmena Linuksovog jezgra (konkretno verzija 2.6.22). U daljem tekstu kada se pristupa trenutno aktivnom procesu, isključivo se koristi makro \texttt{current}.


\subsection{Prosečno vreme izvršavanja}

Prvo je u O(1) planer uključen uticaj prosečnog vremena izvršavanja procesa, što je omogućilo da se prilikom kasnijih pokretanja procesa zna koliko se proces izvršava u proseku i koliko često je bio pokretan.

Ovo je urađeno tako što je u sched.h datoteci zaglavlja unutar Linuksovog izvornog koda dodata heš tabela koja sadrži broj pokretanja određenog koda i prosečno vreme u kojem je isti koristio procesorske resurse. Heš tabela je definisana u exec.c datoteci unutar Linuksovog izvornog koda, a inicijalizacija je urađena u istoj datoteci u funkciji \texttt{do\_execve()}. Da se inicijalizacija ne bi vršila više puta, dodata je pomoćna promenljiva koja ima ulogu zastavice: promenom vrednosti ona sprečava ponovnu inicijalizaciju.

Nakon toga su u kontrolni blok dodata polja u kojima će se pamtiti potrebni podaci iz heš tabele i to su: prosečno vreme izvršavanje kôda, broj pokretanja programa, ključ za heš tabelu  i pokazivač na celu heš tabelu (da se ne bi pravila globalna promenljiva). Da bi se dobio ključ za pristup podacima u heš tabeli, u funkciji \texttt{do\_execve()} je dodat kôd koji otvara izvršnu datoteku, određuje njenu heš vrednost primenom dbj2 algoritma i određuje ključ za heš tabelu (objašnjeno u delu „Heš funkcija i heš tabela”). Kada se dobije ključ, učitaju se vrednosti iz heš tabele i popune se polja u kontrolnom bloku.

Promena prioriteta se vrši u planeru funkcijom \texttt{\_normal\_prio()} koja vraća prioritet modifikovan bonusima/penalima (koji idu u opsegu od -5 do +5). Nakon originalnog kôda za računanje prioriteta, dodat je kôd koji prosečno vreme izvršavanja deli sa brojem pokretanja procesa i dobijenu vrednost koristi da odredi za koliko će uvećati, odnosno umanjiti, već određen bonus originalnim algoritmom. Skala promene prioriteta je određena tako što su uzete najmanja i najveća moguća vrednost koje mogu da se dobiju modifikovanim algoritmom (gledajući opsege tipova promenljivih na 64-bitnom sistemu). Dobijen opseg je izdeljen na sledeći način:


\begin{align*}
    \texttt{odnos} & < 0.47 \rightarrow \texttt{bonus += 5} \\
    0.47 \leq \texttt{odnos} & < 4.7 \rightarrow \texttt{bonus += 3} \\
    4.7 \leq \texttt{odnos} & < 47 \rightarrow \texttt{bonus += 1} \\
    47 \leq \texttt{odnos} & < 470 \rightarrow \texttt{bonus += 0} \\
    470 \leq \texttt{odnos} & < 4700 \rightarrow \texttt{bonus += -1} \\
    4700 \leq \texttt{odnos} & < 47000 \rightarrow \texttt{bonus += -2} \\
    47000 \leq \texttt{odnos} & < 470000 \rightarrow \texttt{bonus += -3} \\
    470000 \leq \texttt{odnos} & < 4700000 \rightarrow \texttt{bonus += -4} \\
    \texttt{odnos} & \geq 4700000 \rightarrow \texttt{bonus += -5} \\
\end{align*}

Izvršena podela nije ravnomerna iz razloga što više testiranih procesa daje vrednost veću od 47, pa je smatrano da je za vrednosti veće od 47 potrebna preciznija skala. Kako se funkcija za promenu prioriteta poziva svaki put kada se proces prebacuje iz niza aktivnih u niz izvršenih procesa, preračunavanje prioriteta se neće izvršiti kada se proces prvi put pokrene.

Ažuriranje podataka u heš tabeli je implementirano u funkciji \texttt{do\_exit()}, koja se poziva kada proces završi svoj rad u potpunosti. Odrađeno je uvećavanje brojača pokretanja programa i preračunavanje novog prosečnog vremena izvršavanja programa. Preračunavanje novog prosečnog vremena se realizuje upotrebom postojeće vrednosti iz heš tabele i \texttt{sched\_time} vrednosti iz kontrolnog bloka.


\subsection{Broj otvorenih datoteka}

Pošto je bitno da proces sa više otvorenih datoteka brže završi rad sa njima da bi ih mogao
prepustiti drugom procesu, kao parametar u originalni planer je dodat i brojač otvorenih
datoteka.

U kontrolni blok su dodata polja za vrednost bonusa i vreme kada je poslednji put izvršeno
prebrojavanje datoteka (potrebno zbog načina implementacije prebrojavanja). Neophodno je
promenljive inicijalizovati na 0 i to se vrši prilikom pokretanja programa u funkciji
\verb|do_execve()|.

Pošto je bitno da proces sa više otvorenih datoteka brže završi rad sa njima da bi ih mogao prepustiti drugom procesu, kao parametar u originalni planer je dodat i brojač otvorenih datoteka.

U kontrolni blok su dodata polja za vrednost bonusa i vreme kada je poslednji put izvršeno prebrojavanje datoteka (potrebno zbog načina implementacije prebrojavanja). Neophodno je promenljive inicijalizovati na 0 i to se vrši prilikom pokretanja programa u funkciji \verb|do_execve()|. 

Prebrojavanje datoteka je implementirano u funkciji \verb|_normal_prio()|, tako što se upotrebom funkcije \verb|files_fdtable()| dobije pokazivač koji pokazuje na listu koja sadrži pokazivače ka podacima otvorenih datoteka. Potrebno je proći kroz ovu listu i prebrojati elemente. Prebrojavanje se vrši tek od 25. sekunde od pokretanja sistema (da se ne bi vršilo prilikom učitavanja sistema). Zbog načina prebrojavanja, odlučeno je da se datoteke ne prebrojavaju prilikom svakog prebacivanja procesa sa niza procesa koji čekaju na niz sa procesima koji su potrošili procesorsko vreme, već na svakih 5s (clock interrupt je uzet u obzir). Proteklo vreme se određuje tako što se od trenutnog vremena (dobijenog \verb|sched_clock()| funkcijom (kernel timekeeping abstractions 2020)) oduzme vreme kada se poslednji put izvršilo prebrojavanje datoteka koje se čuva u dodatoj promenljivoj u kontrolnom bloku.

Sledeći korak je određivanje bonusa na osnovu dobijenog broja otvorenih datoteka (\texttt{ND}). Kako svaki proces ima uvek otvorene bar tri datoteke (stdin, stdout, stderr), bonusi (isključivo) su dodeljivani tek ukoliko proces ima više od pet otvorenih datoteka na sledeći način:

\begin{align*}
    5 \leq \texttt{ND} & < 15 \rightarrow \texttt{bonus += 1} \\
    15 \leq \texttt{ND} & < 25 \rightarrow \texttt{bonus += 2} \\
    25 \leq \texttt{ND} & < 40 \rightarrow \texttt{bonus += 3} \\
    40 \leq \texttt{ND} & < 55 \rightarrow \texttt{bonus += 4} \\ 
    \texttt{ND} & \geq 55 \rightarrow \texttt{bonus += 5}
\end{align*}

Dobijen bonus je potrebno zapamtiti u kontrolnom bloku da bi se mogao ostvariti uticaj na prioritet procesa i kada proces dobije procesorsko vreme, a prebrojavanje datoteka se ne izvršava.

\section{Rezultati i diskusija}

Posebno su urađeni testovi za svaki od dva implementirana algoritama. Svi testovi su pokretani upotrebom Qemu emulatora (qemu, 2021), a vreme je mereno upotrebom izolovanog sata i ključne reči time (Busybox, 2021). Prilikom pokretanja operativnog sistema prvi proces je bio program koji je pokretao test procese. Pored test procesa, operativni sistem je automatski pokretao i procese koji većinu svog izvršavanja spavaju i čekaju poziv. Ovi procesi nisu uklanjani jer ne troše puno procesorskih resursa i smatrano je da ne utiču u dovoljnoj meri na same rezultate.

Za testove sa CFS algoritmom korišćena je verzija Linuksovog jezgra 2.6.23.

\subsection{Prosečno vreme izvršavanja}

Da bi se testirao algoritam koji ima uključen uticaj prosečnog vremena izvršavanja, korišćeno je više skupova procesa različitog tipa. U prvom testu su procesi koji zahtevaju puno procesorskih resursa, dok su u drugom procesi koji veći deo svog izvršavanja provode čekajući poziv. Posmatrano je ukupno prosečno vreme izvršavanja i promena prioriteta procesa u zavisnosti od broja pokretanja samog kôda. Skupovi procesa su pokretani na originalnom O(1) algoritmu, CFS algoritmu i modifikovanom O(1) algoritmu koji uzima u obzir prosečno vreme izvršavanja procesa.

\textbf{Prvi test} je činilo pet programa. Dva programa su programi koji beskonačno rade u pozadini i iziskuju procesorske resurse, a ostala tri su test programi (programA, programB i programC) koji takođe iziskuju procesorske resurse, ali je njihovo prosečno vreme izvršavanja na procesoru iznosilo redom 2.66s, 15.35s i 76.48s. Test programi su pokretani, prvo svaki zasebno, a potom deset puta u grupama od po trideset paralelnih procesa (svaki program se tri puta pokretao) i na kraju opet svaki zasebno. Promena prioriteta za svaki test program se može videti na graficima 1, 2 i 3. 



\Figure{plot_1.pdf}
    {Prioriteti procesa kada je prosečno vreme izvršavanja na procesoru 2.66s (\texttt{programA})}
    {Process priority when average execution time on processor is 2.66s}


\Figure{plot_2.pdf}
    {Prioriteti procesa kada je prosečno vreme izvršavanja na procesoru 15.35s (\texttt{programB})}
    {Process priority when average execution time on processor is 15.35s}

\Figure{plot_3.pdf}
    {Prioriteti procesa kada je prosečno vreme izvršavanja na procesoru 76.48s (\texttt{programC})}
    {Process priority when average execution time on processor is 76.48s}

Program \texttt{programA} je na samom početku dobio prioritet 125, jer je to prioritet koji dodeljuje O(1) planer procesa. Međutim, kada se jednom izvršio, dobijene vrednosti prilikom određivanja prioriteta su ga prebacile u opseg procesa koji dobijaju penal u vrednosti 1, i time mu povećao vrednost prioriteta (smanjio prioritet). Daljim izvršavanjem vrednost prioriteta je opala zato što se broj pokretanja procesa povećavao, dok se prosečno vreme nije značajnije menjalo. Na kraju testa je vrednost prioriteta procesa pala na 124, zbog povećanja broja pokretanja.

Programi \texttt{programB} i \texttt{programC} su iz istih razloga menjali prioritet, samo što su njihova vremena izvršavanja bila veća, pa su stoga upadali u druge opsege. Ukoliko bi se procesi nastavljali izvršavati, njihov prioritet bi opadao dok se ne bi postigla minimalna vrednost. Minimalna vrednost prioriteta je određena najvećom celobrojnom vrednošću na datoj arhitekturi koja predstavlja broj pokretanja programa (broj reprezentovan sa W bajta, gde je W veličina procesorske reči), što bi značilo da je postignut najveći moguć broj pokretanja programa (prosečno vreme se ne menja drastično).

Promena prioriteta je dovela do toga da procesi koji se kraće izvršavaju dobiju veće uzastopne kvante procesorskog vremena bez obzira na to da li se više puta pokreću. Ukoliko se procesi i češće pokreću, ili će dobiti još veće kvante procesorskog vremena, ili će brže steći veću kvantu procesorskog vremena. Prednost je što procesi sa kraćim vremenom izvršavanja dobijaju veću šansu da završe svoj rad u potpunosti. Uticaj na vreme izvršavanja se može videti u sledeća tri grafika, posebno za svaki proces na sva tri algoritma (grafici 4,5 i 6). Iz grafika se može primetiti i da je za sva tri testa vreme izvršavanja bilo kraće u implementiranom nego u CFS algoritmu, ali i nešto duže nego u originalnom O(1) algoritmu. Bitno je napomenuti da se prva i poslednja test iteracija sadržala od serijskog pokretanja programa: \texttt{programA}, \texttt{programB}, \texttt{programC} redom. Ostale interacije su sadržale pokretanja 10 kopija navedenih programa. Posle svake iteracije, izmereno je realno vreme izvršavanja procesa koji je prvi pokrenut. Takođe je bitno napomenuti da se iteracije ne preklapaju.


\Figure{plot_4.pdf}
    {Realno vreme izvršavanja kada je prosečno vreme izvršavanja na procesoru 2.66s.}
    {Real time of process execution when average execution time on processor is 2.66s.}

\Figure{plot_5.pdf}
    {Realno vreme izvršavanja kada je prosečno vreme izvršavanja na procesoru 15.35s.}
    {Real time of process execution when average execution time on processor is 15.35s.}

\Figure{plot_6.pdf}
    {Realno vreme izvršavanja kada je prosečno vreme izvršavanja na procesoru 76.48s.}
    {Real time of process execution when average execution time on processor is 76.48s.}

\textbf{Drugi test} je predstavljao jedan program, \texttt{programD}, čije se izvršavanje svodilo na 5 sekundi spavanja, a potom kratko korišćenje procesorskih resursa radi računanja. Cilj ovog testa je bio da se pokaže kako bi se ponašali procesi koji veći deo izvršavanja provode čekajući zahtev za kratkim izvršavanjem. Test je izvršavan na isti način kao gore-navedeni test sa tom razlikom da umesto tri različita programa, korišćen je samo jedan (\texttt{programD}). U toku celog testa su u pozadini su bila pokrenuta i dva pozadinska procesa (iz prethodnog testa). Bonus, računat u implementiranom algoritmu, je bio najveći moguć već u sedamnaestom pokretanju (treća testna interacija). Vreme za koje se izvršavao testni proces na posmatranim algoritmima se može videti na sledećem grafiku (grafik 7).


\Figure{plot_7.pdf}
    {Realno vreme izvršavanja procesa po iteraciji, test program \texttt{programD}.}
    {Real time of process execution for test program \texttt{programD}.}

Sa grafika 7 se može videti da su procesi imali kraće vreme izvršavanja u implemetiranom algoritmu, nego u originalnom O(1) kada ih je bilo više paralelno pokretnutih. Između CFS algoritma i implementiranog se ne vidi neka značajna razlika za ovakav test.

\subsection{Broj otvorenih datoteka}

Testiranje algoritma koji ima uključen brojač otvorenih datoteka u određivanje prioriteta procesa se vršilo pokretanjem procesa koji su otvarali različit broj datoteka. Vreme izvršavanje nije uzeto u obzir, već se posmatrala samo promena dinamičkog prioriteta procesa. Modifikovani algoritam je upoređen samo sa O(1) planerom, pošto dinamički prioritet ne postoji u CFS algoritmu.

Pokrenut je skup procesa A, B, C, D, E koji redom otvaraju 2, 12, 22, 37 i 52 datoteke. Nakon toga, procesi obavljaju izračunavanje koje kratko iziskuje procesorske resurse, da bi se mogao promeniti i očitati dinamički prioritet. Razliku u dinamičkom prioritetu možemo videti na grafiku 8.

\Figure{plot_8.pdf}
    {Realno vreme izvršavanja kada je prosečno vreme izvršavanja na procesoru 76.48s.}
    {Real time of process execution when average execution time on processor is 76.48s.}

Ideja je bila da se izričito poveća prioritet procesima koji imaju više otvorenih datoteka, da bi isti što pre završili svoj rad i oslobodili drugim procesima pristup datotekama. Sa grafika se može videti da je željeni efekat postignut.


\section{Zaključak}

Linuksova implementacija O(1) planera procesa je modifikovana na dva načina. Prva modifikacija je imala cilj da procesi koji se često pokreću i/ili kratko traju dobiju veći prioritet. Druga modifikacija je imala cilj da i broj otvorenih datoteka utiče na prioritet procesa. Rezultati su pokazali da se za obe modifikacije postigao željeni efekat.

Prva modifikacija je omogućila da često pokretani procesi dobiju veći prioritet. Ovaj efekat je isto postignut i za procese koji se kratko izvršavaju. Ovakvom implementacijom je produženo vreme pokretanja procesa, iz razloga što se prilikom svakog pokretanja vrši heširanje celog izvršnog koda i postoji dodatno zauzeće memorije u vidu heš tabele. Implementirani algoritam je omogućio kraće izvršavanje procesa koji iziskuju dosta procesorskih resursa u odnosu na CFS algoritam. U poređenju sa O(1) algoritamom, implementirani algoritam je pokazao slične rezltate za vreme izvršavanje procesa. Razlog za ovakve rezultate je što CFS algoritam pokušava da uspostavi jednakost između procesa. Ovim on pokušava da ostvari jednakost između pozadinskih i test procesa. Nešto duže izvršavanje u poređenju sa O(1) algoritmom govori da dati bonusi i penali za ovakve testove nemaju dovoljan uticaj da bi se nadmašilo gubljenje vremena na određivanje heš vrednosti koda. Implementirani algoritam je pokazao bolje rezultate u odnosu na O(1) algoritam u slučaju procesa koji čekaju zahtev za izvršavanje. Razlog za ovakav rezultat je što algoritam daje veći prioritet procesu, jer je očekivano da se kraće izvršava, što će procesu omogućiti veću kvanut procesorskog vremena u odnosu na pozadinske procese. U poređenju sa CFS  agloritmom, dobijeni rezultati su bili slični ili lošiji. Rezultat može da se objasni činjenicom da će kod CFS algoritma proces imati jako malo potrošenog proceskorskog vremena (sa uračunatim bonusima i penalima). Iz ovog razloga će dobijati vrlo brzo i dovoljno dugo procesorske resurse da završi svoj rad bez čestog ponovnog čekanja.

Druga modifikacija je omogućila povećanje prioriteta procesa sa povećanjem broja otvorenih datoteka. Kako prebrojavanje iziskuje prolazak kroz listu svih otvorenih datoteka, ono se vršilo na svakih 5 sekundi, ili ređe. Dobijenim rezltatima je pokazano da je željeni efekat postignut. Odlučeno je da se ne daju penali, iz razloga što je smatrano da i u slučaju da proces ima makar jedanu otvorenu datoteku treba što pre da oslobodi drugim procesima pristup datoteci. Ovakva modifikacija algoritma bi mogla da ima veću prednost u sistemima gde više procesa zahteva pristup istoj datoteci u isto vreme. 

Moguće poboljšanje za planer koji uzima broj otvorenih datoteka u obzir bi bilo efikasnije brojanje otvorenih datoteka. Umesto brojanja elementata u listi, modifikacija strukture otvorenih datoteka bi dovela do boljih performansi. Modifikacija bi se mogla izvršiti jednostavnim dodavanjem brojača. Ovim bi se izgubila potreba za prolaskom kroz listu svih otvorenih datoteka, pa bi ovakva modifikacija dovela do kraćeg vremena izvršavanja planera procesa u odnosu na trenutno implementirani.

Takođe, nije provereno da li bi sistem sa ovakvim dodatnim zauzećem memorije mogao da pokrene isti broj procesa, kao i originalni bez dodatnih modifikacija.

\section{Literatura}
\Source{%
    Algorithms in C, Hash algorithms,\\
    https://thealgorithms.github.io/C/d7/d3b/\\group\_\_hash.html
    [Pristupljeno 12. maja 2021].
}
\Source{%
    Busybox. Dostupno na:\\
    https://www.commandlinux.com/man-page/man1/\\busybox.1.html
    [Pristupljeno 12. maja 2021].
}
\Source{%
    Clock sources, Clock events, sched clock() and delay timers, (kernel timekeeping abstractions). 
    Dostupno na:\\ https://www.kernel.org/doc/Documentation/\\timers/timekeeping.txt 
    [Pristupljeno 3. septembra 2020].
}
\Source{%
    Cormen T. H., Leiserson C. E., Rivest R. R., Stein C. 2009. 
    \emph{Introduction to Algorithms}.
    Kembridž: MIT Press
}
\Source{%
    Ishkov N. 2015.
    \emph{A complete guide to Linux process scheduling}. MSc Thesis.
    University of Tampere, School of Information Sciences
}
\Source{%
    Knuth D. E. 1998.
    \emph{The art of computer programming}.
    Boston: Addison-Wesley
}
\Source{%
    Linux Interview Questions 2020. Linux Interview Questions: Open Files / Open File Descriptors.
    Dostupno na:\\https://www.thegeekdiary.com/linux-interview-\\questions-open-files-open-file-descriptors/
    [Pristupljeno 3. septembra 2021].
}
\Source{%
    Linux Programmer’s manual, stdin(03). 
    Dostupno na:\\https://man7.org/linux/man-pages/man3/stdout.3\\.html 
    [Pristupljeno 12. maja 2021].
}
\Source{%
    Love R. 2005.
    \emph{Linux kernel development, second edition}.
    Karmel: Sams Publishing
}
\Source{%
    Love R. 2010.
    \emph{Linux kernel development, third edition}.
    Boston: Addison-Wesley
}
\Source{%
    Marić M. 2017.
    \emph{Operativni sistemi}.
    Beograd: Univerzitet u Beogradu
}
\Source{%
    Qemu.
    Dostupno na: https://qemu.org 
    [Pristupljeno 3. septembra 2020].
}
\Source{%
    Schmidt B. 2015.
    CPU Clocks and Clock Interrupts, and Their Effects on Schedulers.
    U \emph{Overload}, \textbf{23} (130): 23.
    Dostupno na:\\https://accu.org/journals/overload/23/130/\\schmidt\_2185/
}
\Source{%
    Šantavec D. 2019.
    Implementacija kružnog algoritma sa zadržavanjem u planeru procesa na MP/M II operativnom sistemu.
    \emph{Petničke sveske}, 78: 329.
}
\Source{%
    Tanenbaum S. A., Bos H. 2015.
    \emph{Modern Operating Systems}.
    London: Pearson
}
\Source{%
    tuxthink 2012. \emph{Module to print the open files of a process}.
    Dostupno na: \\https://tuxthink.blogspot.com/2012/05/module-to-\\print-open-files-of-process.html
    [Pristupljeno 3. septembra 2020.]
}
\EndPaper
\clearpage