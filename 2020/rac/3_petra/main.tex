\Title{Predviđanje rezultata izbora u realnom vremenu algoritmom višestruke linearne regresije pomoću analize sentimenta tvitova}
\TitleEng{Real-time election results prediction using Multiple Linear Regression algorithm and tweet sentiment analysis}
\Author{Petra Avramović}
\begin{Abstract}
U ovom radu je ispitana upotreba analize sentimenta tvitova i algoritma višestruke linearne regresije za razvijanje statističkog modela koji bi predviđao rezultate izbora u realnom vremenu. Za analizu sentimenta je korišćen VADER, alat za generalnu analizu sentimenta baziran na leksikonu. Razvijeno je četiri modela: dva pomoću algoritma višestruke linearne regresije u kojoj su nezavisne promenljive bile prosečni sentiment i ukupan broj tvitova za svakog kandidata i dva pomoću jednostavne linearne regresije u kojoj je jedina nezavisna promenljiva bila ukupan broj tvitova. Modeli se, takođe, razlikuju i po kvantitetu tvitova upotrebljenih za njihovu obuku i testiranje. Dva koriste tvitove objavljene u toku jednog meseca pred dan izbora, a druga dva tvitove objavljene u toku jedne nedelje pred dan izbora. Rezultati testiranja pokazuju da sentiment tvitova u većini slučajeva nema značajan uticaj na predikcije modela, ali da period u kome su objavljeni tvitovi na osnovu kojih se vršilo predviđanje ima. Najbolje rezultate je postigao model koji koristi samo ukupan broj tvitova objavljenih u toku jedne nedelje pred dan izbora, za koji Pirsonov koeficijent korelacije između pravih i predviđenih vrednosti iznosi 0.86, a koren srednjeg kvadratnog odstupanja (RMSE) između predviđenih i ostvarenih rezultata iznosi 0.07.
\end{Abstract}
\begin{AuthorEx}
Petra Avramović, Sokobanja, 3. razred Srednje škole “Branislav Nušić” u Sokobanji

MENTOR:\\
Stefan Nožinić, inžinjer za razvoj softvera, Continental Automotive d.o.o.
\end{AuthorEx}
\begin{AbstractEng}
This paper investigates the use of sentiment analysis of tweets and linear regression algorithm for election results prediction. Four distinct models were developed in order to determine whether sentiment of tweets is a predictive feature and also to determine the optimal time frame for tweet extraction preceding the election. The data set consisted of 40 million tweets concerning 6 different elections which were extracted during a time period of one month before the election day. Sentiment analysis was done using VADER (Valence Aware Dictionary for sEntiment Reasoning), a simple rule-based model for general sentiment analysis. VADER has proven to be especially effective in analyzing social media text. Two of the developed models, named model 1 and model 2, were implemented using multiple linear regression where the independent variables were the total number of tweets and the average sentiment score for each candidate. The other two model, named model 3 and model 4, were implemented using simple linear regression where the independent variable was the total number of tweets for each candidate. The dependent variables for all 4 models were the actual election results quantified as the share of votes each candidate had won. For models 1 and 3 all of the collected tweets were used, while for models 2 and 4 only the ones published in a time period of a week preceding the election were used. This approach yielded mixed results. The results imply that the sentiment of tweets doesn’t have a significant impact on the predictions, but that the time frame in which tweets were published in does. The best preforming model was model 4 for which correlation coefficient between predicted and actual results was 0.86 and RMSE (Root Mean Square Error) was 0.07 which suggests that the optimal time frame for election results prediction is one week before the election day.
\end{AbstractEng}

\StartDoublePaper
\label{rac.petra}

\section{Uvod}

Zaključno sa drugim kvartalom 2020. godine, Tviter ima preko 186 miliona aktivnih korisnika koji svakodnevno dele svoje mišljenje o raznim događajima pomoću kratkih poruka koje se zovu tvitovi (Statista 2020). Tvitovi mogu da sadrže najviše 280 karaktera, pa zato predstavljaju izuzetan medij za koncizno izražavanje stavova o nekoj temi. Analiza sentimenta tvitova može dati značajan uvid u mišljenje javnosti. Sve ovo čini Tviter vrlo pogodnim za predviđanje ishoda izbora (Kursuncu et al. 2019). 

\AuthorExHere

Cilj ovog rada je kreiranje statističkog modela koji bi omogućio predviđanje rezultata izbora u realnom vremenu, što podrazumeva da je predikcije moguće vršiti brzo i efikasno. U referentnom radu (Ramteke et al. 2016) za analizu sentimenta korišćeni su algoritmi nadgledanog mašinskog učenja: polinomni naivni Bajesov klasifikator (Multinomial Naive Bayes) i metod potpornih vektora (Support Vector Machine), pri čemu je F1-mera za pouzdanost (Hosin i Sulaiman 2015) iznosila 0.94 i 0.97, respektivno. Iako daje prilično precizne rezultate, ovakav pristup zahteva treniranje modela za analizu sentimenta, što bi onemogućilo implementaciju u realnom vremenu. Takođe, ovako treniran model bi bio specifičan samo za izbore na kojima je treniran, pa bi za primenu na druge izbore morao biti treniran iznova. Stoga, ovaj pristup ne bi mogao da se koristi za razvijanje univerzalnog modela. Za takav model je bolje koristiti VADER (Valence Aware Dictionary for sEntiment Reasoning; Hutto i Gilbert 2014). VADER je alat za analizu sentimenta koji je baziran na leksikonu, što podrazumeva da se može koristiti na bilo kojem skupu podataka bez prethodnog treniranja. Takođe, VADER postiže vrlo precizne rezultate pri analizi tvitova (F1 = 0.96; Hutto i Gilbert 2014). Još jedna prednost upotrebe ovog alata je to što, za razliku od modela treniranih mašinskim učenjem, VADER određuje intenzitet sentimenta brže i efikasnije (Hutto i Gilbert 2014). S druge strane, u jednom od referentnih radova (Gaurav et al. 2013) za predviđanje izbora korišćen je isključivo ukupan broj tvitova za svakog kandidata. Ovakav pristup je dao vrlo precizne rezultate i u najboljem slučaju koeficijent korelacije iznosio je 1 a koren srednjeg kvadratnog odstupanja između predviđenih i ostvarenih rezultata 0, dok je u najgorem slučaju koeficijent korelacije iznosi 0.4, a koren srednjeg kvadratnog odstupanja 0.19. U referentnom radu (Zolghadr et al. 2018) upoređene su tri metode za predviđanje rezultata izbora na osnovu određenih političkih i ekonomskih varijabli: regresija potpornih vektora, veštačke neuronske mreže i linearna regresija. Regresija potpornih vektora dala je najbolje rezultate za koju koren srednjeg kvadratnog odstupanja između predviđenih i ostvarenih rezultata iznosi 0.01, dok za veštačke neuronske mreže i linearnu regresiju ta metrika iznosi 0.013 i 0.104, respektivno. 

\section{Metod}

\subsection{Podaci}

Tvitovi su prikupljeni za šest izbora u državama engleskog govornog područja: predsedničke izbore u SAD 2016. i 2012. godine, federalne izbore u Kanadi 2019. i 2015. godine i parlamentarne izbori u Velikoj Britaniji 2019. i 2017. godine. Skup podataka se sastojao od 40,950,480 tvitova preuzetih iz vremenskog perioda od jednog meseca pred dan izbora (tabela 1). Prikupljanje podataka vršeno je pomoću Pajton biblioteke \emph{getoldtweets3}.

\TableDouble{
    \begin{tabular}{l|l|l}
        Izbori & Broj tvitova & Vremenski period \\
        \hline
        SAD 2016. & 21,108,680 & 8. 10. -- 8. 11. 2016. \\
        SAD 2012. & 13,889,488 & 6. 10. -- 6. 11. 2012. \\
        Kanada 2019. & 1,077,132 & 21. 9. -- 21. 10. 2019 \\
        Kanada 2015. & 615,685 & 19. 9. -- 19. 10. 2015. \\
        UK 2019. & 2,687,302 & 12. 11. -- 12. 12. 2019. \\
        UK 2017. & 1,572,193 & 8. 5. -- 8. 6. 2017. \\
    \end{tabular}
}{Skup podataka.}{Data set.}

\subsection{Priprema podataka}

Tvitovi su minimalno procesirani pre analize sentimenta kako nijanse u izražavanju specifične za kontekst društvenih mreža ne bi bile izgubljene. Stoga, priprema podataka podrazumevala je otklanjanje specijalnih karaktera, poput simbola @, i linkova iz tvitova, jer mogu negativno da utiču na analizu sentimenta (Ramteke et al. 2016). 

\subsection{Klasifikacija}

Klasifikacija tvitova je izvršena na osnovu ključnih reči koje su određene za svakog kandidata zasebno. U ključne reči spadaju ime i prezime kandidata i politička partija koju taj kandidat predstavlja. Klasifikacija je automatizovana tako što je za svaki tvit provereno prisustvo ključnih reči. Ako je neka od njih prisutna u tekstu tvita, onda je taj tvit sačuvan u poseban CSV fajl specifičan za kandidata čija je ključna reč detektovana. Na taj način uklonjeni su svi prazni tvitovi i tvitovi u kojima se ne spominje nijedan kandidat, jer takvi tvitovi nisu smatrani relevantnim za izbore.

\subsection{Analiza sentimenta}

VADER kao ulazni parametar uzima tekst i računa rezultat za četiri kategorije: pozitivnu, negativnu, neutralnu komponentu teksta, kao i generalni polaritet teksta. Prve tri kategorije predstavljaju odnos pozitivnih, odnosno negativnih ili neutralnih leksičkih predmeta (reči, emotikona, inicijalizama) i svih leksičkih predmeta u određenom tekstu. Generalni polaritet teksta predstavlja normalizovanu, kompozitnu vrednost sentimenta teksta i umnogome zavisi od modifikatora intenziteta sentimenta. Modifikatori intenziteta sentimenta predstavljaju skup gramatičkih i sintaksičkih pravila koja ukazuju na promenu intenziteta sentimenta. U modifikatore spadaju znakovi interpunkcije, kapitalizacija reči, prilozi, suprotni veznik „ali“ i negacija. Takođe, VADER u obzir uzima i akronime, inicijalizme, emotikone i žargonske izraze koji se često javljaju u tvitovima zbog ograničenog broja karaktera. Brojna vrednost polariteta teksta izračunava se sumiranjem vrednosti sentimenta svake reči u tekstu uz primenu modifikatora intenziteta sentimenta i kreće se između –1 i 1, gde –1 predstavlja najnegativniju vrednost sentimenta, a 1 najpozitivniju (Hutto i Gilbert 2014). U ovom radu je, od četiri navedene kategorije, korišćen samo generalni polaritet teksta.

\section{Model}

\subsection{Promenljive}

Pre modelovanja, ukupan broj tvitova svakog kandidata se normalizuje formulom:
\begin{equation*}
    \Bar{C}_{ik}=\frac{\sum^{k}_{d=1} C_{id}}{\sum^N_{d=1}\left(\sum^k_{d=1}C_{id}\right)}
\end{equation*}
gde $\Bar{C}_{ik}$ predstavlja normalizovani ukupan broj tvitova za kandidata $i$ u rasponu od $k$ dana, $C_{id}$ ukupan broj tvitova za kandidata $i$ na dan $d$, $N$ broj kandidata, a $k$ ukupan broj dana.

Nakon kvantifikacije sentimenata tvitova određen je prosečni sentiment za svakog kandidata. Prosečni sentiment se računa po formuli:
\begin{equation*}
    \Bar{s}_{ik}=\frac{\sum^{C_{ik}}_{t=1} s_t}{C_{ik}}
\end{equation*}
gde $\Bar{s}_{ik}$ predstavlja prosečni sentiment za kandidata $i$, $s_t$ brojnu vrednost sentimenta za tvit $t$, a $C_{ik}$ ukupan broj tvitova za kandidata $i$ u rasponu od $k$ dana.

\subsection{Algoritam}

Kao algoritam odabrana je linearna regresija zbog efikasne i lake implementacije i zbog toga što je zapažena približno linearna veza između promenljivih koje su učestvovale u regresionoj analizi. Linearna regresija predstavlja modelovanje relacija između jedne ili više zavisnih promenljivih i jedne ili više nezavisnih promenljivih. Opšta jednačina linearnog modela za neku instancu $i$ glasi:
\begin{equation*}
    y_i = \beta_0 + \beta_1 x_{i1} + \cdots + \beta_1 x_{ip} + \epsilon_i \quad i = 1, \dots, n
\end{equation*}

Ovih $n$ jednačina mogu se složiti u vektorski oblik kao
\begin{equation*}
    Y = X\beta + \epsilon
\end{equation*}
gde je $Y$ vektor izmerenih vrednosti koji se naziva zavisna promenljiva, $X$ matrica n-dimenzionalnih kolonskih-vektora $X_j$, poznatih kao nezavisne promenljive, $\beta$ vektor koeficijenata, a $\epsilon$ vektor reziduala. Reziduali, tj. greške predviđanja predstavljaju razliku između stvarne vrednosti zavisne promenljive ($y_i$) i predviđene vrednosti zavisne promenljive ($\hat{y}_i$). Predviđanja se vrše pomoću formule:
\begin{equation*}
    \hat{y}_i = \beta_0 + \beta_1 x_{i1} + \cdots + \beta_1 x_{ip} \quad i = 1, \dots, n
\end{equation*}
pa je odatle $\epsilon_i = y_i - \hat{y}_i$.

Optimalni koeficijenti dobijaju se metodom najmanjih kvadrata, tako što se minimizuje suma kvadrata reziduala (Taboga 2017).
\begin{equation*}
    \beta = \argmin_{\beta_0,\dots,\beta_p} \sum^n_{i=1} \left(
        y^{(i)} - \left(
            \beta_0 + \sum^p_{j=1} \beta_j x^{(i)}_j
        \right)
    \right)^2
\end{equation*}

U ovom radu razvijeno je četiri modela. Za model 1 i model 3 korišćeni su svi prikupljeni tvitovi, a za model 2 i model 4 samo tvitovi objavljeni u toku jedne nedelje pred dan izbora. Takođe, u modelima 1 i 2 primenjena je višestruka linearna regresija gde su nezavisne promenljive bile normalizovani ukupan broj tvitova i prosečni sentiment, dok je u modelima 3 i 4 primenjena jednostavna linearna regresija gde je nezavisna promenljiva bila normalizovani ukupan broj tvitova. Zavisne promenljive za sva četiri modela bile su stvarni rezultati izbora kvantifikovani kao udeo glasova koji je kandidat osvojio.

Jednačina modela 1 i 2 glasi:
\begin{equation*}
    y_i = \beta_0 + \beta_1 \Bar{C}_{ik} + \beta_2 \Bar{s}_{ik} + \epsilon_i
\end{equation*}

Jednačina modela 3 i 4 glasi:
\begin{equation*}
    y_i = \beta_0 + \beta_1 \Bar{C}_{ik} + \epsilon_i
\end{equation*}

Gde $y_i$ predstavlja stvarne rezultate izbora za kandidata $i$, $\beta_1$ i $\beta_2$ koeficijente, $\beta_0$ odsečak na y-osi, $\Bar{C}_{ik}$ normalizovani ukupan broj tvitova za kandidata $i$ u rasponu od k dana, a $\Bar{s}_{ik}$ prosečni sentiment (važi da je za model 1 i 3 $k = 30$, a za model 2 i 4 $k = 7$).

\subsection{Normalizacija rezultata}

Nakon što model predvidi rezultate, potrebno je normalizovati ih za svaki pojedinačan izbor kako bi bili izraženi kao udeo glasova koji je kandidat osvojio. Rezultati su normalizovani formulom:
\begin{equation*}
    \Tilde{y}_i = \hat{y}_i + \frac{1 - \sum^N_{i=1}\hat{y}_i}{N}
\end{equation*}
gde $\Tilde{y}_i$ predstavlja normalizovane rezultate za kandidata $i$, $\hat{y}_i$ prognozirane rezultate za kandidata $i$ i $N$ broj kandidata koji učestvuje u izboru za koji se normalizuju rezultati.

\section{Evaluacija}

Metrike korišćene za evaluaciju modela su: koeficijent korelacije i standardna greška definisana kao koren sume kvadrata odstupanja između pravih i predviđenih vrednosti.

\subsection{Koeficijent korelacije}

Koeficijent korelacije između stvarnih i normalizovanih prognoziranih rezultata računa se po formuli:
\begin{equation*}
    r = \frac{
        \sum^N_{i=1}(\Tilde{y}_i - \Tilde{y})(y_i - \Bar{y})
    }{
        \sqrt{\sum^N_{i=1}(\Tilde{y}_i - \Tilde{y})^2\sum^N_{i=1}(y_i - \Bar{y})^2}
    }
\end{equation*}
gde $r$ predstavlja koeficijent korelacije između stvarnih i normalizovanih prognoziranih rezultata, $\Tilde{y}_i$
normalizovani prognozirani rezultat za kandidata $i$, $\Tilde{y}$ srednju vrednost normalizovanih prognoziranih
rezultata, $y_i$ stvarni rezultat za kandidata $i$, $\Bar{y}$ srednju vrednost stvarnih rezultata, a $N$ broj kandidata.

\subsection{Koren srednjeg kvadratnog odstupanja (RMSE)}

Koren srednjeg kvadratnog odstupanja (RMSE) računa se po formuli:
\begin{equation*}
    RMSE = \sqrt{\frac{\sum^N_{i=1}(y_i - \Tilde{y}_i)^2}{N}}
\end{equation*}
gde $\Tilde{y}$ predstavlja normalizovane prognozirane rezultate za kandidata $i$, $y_i$ stvarne rezultate za
kandidata $i$ i $N$ broj kandidata.

\section{Rezultati i diskusija}

Kako bi se utvrdila uspešnost modela u predviđanju, nezavisno od skupa podataka za obuku i skupa za testiranje, urađena je k-struka unakrsna validacija (Arlot i Celisse 2009). Skup podataka je particionisan na 6 disjunktnih podskupova (k1, k2, …, k6), gde svako k predstavlja uzorak tvitova koji su se odnosili na tačno jedan događaj izbora. Proces obuke i testiranja modela ponovljen je 6 puta. Svaki put, jedan podskup korišćen je za testiranje modela, a ostalih 5 za obuku. U svakoj iteraciji unakrsne validacije izmereni su koeficijent korelacije i koren srednjeg kvadratnog odstupanja između ostvarenih i predviđenih rezultata, a na kraju je određena njihova prosečna vrednost, što je prikazano na slikama 9 i 10.

\Table{
    \begin{tabular}{l|l|r}
        Izbor & Kandidat & Indeks \\
        \hline
        \multirow{2}{6em}{SAD 2016.} & Hilari Klinton & 1 \\
        & Donald Tramp & 2 \\
        \hline
        \multirow{2}{6em}{SAD 2012.} & Barak Obama & 3 \\
        & Mit Romni & 4 \\
        \hline
        \multirow{6}{6em}{Kanada 2019.} & Džastin Trudo & 5 \\
        & Endru Šir & 6 \\
        & Džagmit Sing & 7 \\
        & Elizabet Mej & 8 \\
        & Maksim Bernie & 9 \\
        & Iv-Fransoa Blanše & 10 \\
        \hline
        \multirow{5}{6em}{Kanada 2015.} & Džastin Trudo & 11 \\
        & Stiven Harper & 12 \\
        & Elizabet Mej & 13 \\
        & Tomas Mulker & 14 \\
        & Žil Dusep & 15 \\
        \hline
        \multirow{4}{6em}{UK 2019.} & Boris Džonson & 16 \\
        & Džeremi Korbin & 17 \\
        & Nikola Sturdžon & 18 \\
        & Džo Svinson & 19 \\
        \hline
        \multirow{6}{6em}{UK 2017.} & Tereza Mej & 20 \\
        & Džeremi Korbin & 21 \\
        & Nikola Sturdžon & 22 \\
        & Tim Feron & 23 \\
        & Geri Adams & 24 \\
        & Arlin Foster & 25 \\
    \end{tabular}
}{Indeksi kandidata.}{Candidate indices.}

\Figure{plot_1.pdf}
    {Normalizovan ukupan broj tvitova za svakog kanidata.}
    {Normalized total number of tweets for each candidate (week, month).}

\Figure{plot_2.pdf}
    {Prosečni sentiment za svakog kandidata.}
    {Average sentiment for each candidate (week, month).}

\Figure{plot_3.pdf}
    {Ostvareni i predviđeni rezultati za predsedničke izbore u SAD-u 2016.}
    {Actual and predicted results for the 2016 US presidential election.}

\Figure{plot_4.pdf}
    {Ostvareni i predviđeni rezultati za predsedničke izbore u SAD-u 2012.}
    {Actual and predicted results for the 2012 US presidential election.}

\Figure{plot_5.pdf}
    {Ostvareni i predviđeni rezultati za federalne izbore u Kanadi 2019.}
    {Actual and predicted results for the 2019 Canada federal election.}

\Figure{plot_6.pdf}
    {Ostvareni i predviđeni rezultati za federalne izbore u Kanadi 2015.}
    {Actual and predicted results for the 2015 Canada federal election.}

\Figure{plot_7.pdf}
    {Ostvareni i predviđeni rezultati za parlamentarne izbore u UK 2019.}
    {Actual and predicted results for the 2019 UK general election.}

\Figure{plot_8.pdf}
    {Ostvareni i predviđeni rezultati za parlamentarne izbore u UK 2015.}
    {Actual and predicted results for the 2015 UK general election.}

Dobijeni rezultati pokazuju da upotreba tvitova objavljenih tokom jedne nedelje pred dan izbora daje bolje rezultate nego upotreba tvitova objavljenih tokom jednog meseca pred dan izbora, što se najbolje vidi na primeru predsedničkih izbora 2016. u Sjedinjenim Američkim Državama (slika 10). Poređenje rezultata modela 1 i 2 sa rezultatima modela 3 i 4 pokazuje da sentiment tvitova nema značajan uticaj na krajnje rezultate osim u slučaju parlamentarnih izbora 2017. u Velikoj Britaniji, gde je jedino model 1 tačno predvideo pobednika (slika 8). Najveći nedostatak ovih modela primećuje se na parlamentarnim izborima jer su previše naklonjeni kandidatima koji su zapravo osvojili vrlo mali deo glasova (slika 5, 6, 7, i 8). Najbolji rezultat je postigao model 4 za koji prosečni koeficijent korelacije između predviđenih i ostvarenih rezultata iznosi 0.86 (slika 9), a prosečni koren srednjeg kvadratnog odstupanja između predviđenih i ostvarenih rezultata 0.07 (slika 10). 


\Figure{plot_9.pdf}
    {Koeficijent korelacije između predviđenih i ostvarenih rezultata.}
    {Correlation coefficient between predicted and actual results.}

\Figure{plot_10.pdf}
    {RMSE između predviđenih i ostvarenih rezultata.}
    {RMSE between predicted and actual results.}

\section{Zaključak}

Ovaj rad je pokazao da kombinacija ovih modela i podataka sa Tvitera ne može da vrši pouzdano predviđanje rezultata izbora. Uprkos tome, predloženi modeli su postigli zadovoljavajuće rezultate u poređenju sa modelima linearne regresije iz referentnog rada (Zolghadr et al. 2018) i mogli bi dalje biti poboljšani primenom karakteristika nevezanih za Tviter poput biografskih podataka o kandidatima i političkih i ekonomskih varijabli koje mogu imati uticaj na glasače. Takođe, u narednim radovima treba razviti posebne modele za predsedničke i parlamentarne izbore zbog njihove različite prirode. Upotreba linearne regresije je pokazala dobre rezultate za predsedničke izbore, dok bi za parlamentarne bolje bilo primeniti eksponencijalnu regresiju zbog velike razlike u udelu osvojenih glasova među kandidatima, koju je nemoguće linearno modelovati sa velikom preciznošću.

\section{Literatura}

\Source{%
    Arlot S., Celisse C. 2009.
    A survey of cross-validation procedures for model selection.
    \textit{Statistics Surveys}, 4
}
\Source{%
    Gaurav M., Kumar A., Srivastava A., Miller S. 2013.
    Leveraging Candidate Popularity on Twitter to Predict Election Outcome. 
    U \textit{SNAKDD '13: Proceedings of the 7th Workshop on Social Network Mining and Analysis.}
    New York: ACM, članak 7, str. 1-8.
}
\Source{%
    Hossin M., Sulaiman, M.N. 2015.
    A review on evaluation metrics for data classification evaluations.
    U \textit{International Journal of Data Mining \& Knowledge Management Process}, str. 1.
}
\Source{%
    Hutto C. J., Gilbert E. 2014. 
    VADER: A Parsimonious Rule-based Model for Sentiment Analysis of Social Media Text.
    U \textit{Proceedings of the 8th International AAAI Conference on Weblogs and Social Media (ICWSM-14)} (ur. E. Adar i P. Resnick).
    Palo Alto: AAAI Press (Association for the Advancement of Artificial Intelligence), str. 216-25.
}
\Source{%
    Kursuncu U., Gaur M., Lokala U., Thirunarayan K., Sheth A., Arpinar I. 2019. Predictive Analysis
    on Twitter: Techniques and Applications. U \textit{Emerging Research Challenges and Opportunities in
    Computational Social Network Analysis and Mining}, str. 67-104.
}
\Source{%
    Ramteke J., Shah S., Godhia D., Shaikh A. 2016. Election result prediction using Twitter sentiment
    analysis. U \textit{2016 International Conference on Inventive Computation Technologies (ICICT)}.
    IEEE, str. 1-5.
}
\Source{%
    Statista. 2020. \textit{Number of monetizable daily active Twitter users (mDAU) worldwide.}
    Dostupno na:\\https://www.statista.com/statistics/970920/\\monetizable-daily-active-twitter-users-worldwide/
}
\Source{%
    Taboga M. 2017. \textit{Linear regression models, Lectures on probability theory and mathematical
    statistics, Third edition}. Kindle Direct Publishing.
}
\Source{%
    Zolghadr, M., Niaki S. A. A., Niaki, S. T. A. 2018. Modeling and forecasting US presidential election
    using learning algorithms. \textit{Journal of Industrial Engineering International}.
    Heidelberg: Springer, \textbf{14} (3): 491-500.
}

\vspace{2cm}

\EndPaper